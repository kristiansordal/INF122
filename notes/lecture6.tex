\documentclass{article}
\input{preamble.tex}

\begin{document}
    \section{Forelesning 7}

    \subsection{Plan for forelesningen}
    \begin{itemize}
        \item Let-bindinger
        \item Introduksjon til rekursjon
    \end{itemize}

    \subsubsection{Let-bindinger}

    Vi kan innføre midlertidige verdier i programmene våre ved å bruke nøkkelordet \texttt{let} og \texttt{in}:

    \begin{lstlisting}
        solveQuadratic a b c
            = let discriminant = b*b - 4*a*c 
                in ((-b + sqrt discriminant)/(2*a)
                    , (-b - sqrt discriminant)/(2*a))
    \end{lstlisting}

    \subsubsection{Mønster matching}
    Hva gjør denne funksjonen?
    \begin{lstlisting}
        ordpair :: (Ord a) => a -> a -> (a,a)
        ordpair x y = if x <= y then (x, y) else (y, x)
    \end{lstlisting}

    Den plasserer det minste elementet i input i første index i tuplen som funksjonen returnerer.
    \medskip

    Hva gjør funksjonen \texttt{diff}

    \begin{lstlisting}
        diff x y = let (a, b) = ordpair x y
            in (b - a)
    \end{lstlisting}


    \subsubsection{Flere let features}

    Let uttrykk kan brukes
    \begin{itemize}
        \item til å definerer funksjoner
        \item i \( do \)-notasjon
    \end{itemize}


    \subsection{Rekursjon}
    Fra engelsk: "recursion" fra

    \begin{itemize}
        \item "recur" på norsk "å skje igjen" eller "gjentatt"
    \end{itemize}

    (egentlig fra latin "recurro", som betyr "å løpe tilbake"), men hva er det som "gjentas" eller skjer igjen?

    \begin{definition}
        En funksjon er rekursiv dersom den kaller seg selv igjen (muligens med nye parametere).
    \end{definition}    
    \newpage

    \begin{eg}
        Her er en av ukesoppgavene fra uke 1, skrevet rekursivt.
        \begin{lstlisting}
            triangle :: Integer -> Integer
            triangle n = if n == 0
                then 0
                else n + (triangle (n - 1))
        \end{lstlisting}
    \end{eg}

    \subsubsection{Oppsett}

    Strukturen i en typisk rekursiv funksjon

    \begin{itemize}
        \item Grunntilfelle(r)
        \item Rekursive tilfeller
    \end{itemize}

    \begin{eg}
        Vi kan la mønstrene i argumentet bestemme grunntilfellet og rekursive tilfeller

        \begin{lstlisting}
            sum' :: [Integer] -> Integer
            sum' [] = 0
            sum' (x:xs) = x + sum' xs
        \end{lstlisting}
    \end{eg}

    \begin{eg}
        Rekursive funskjoner er relativt enkle å bevise ting om

        \begin{lstlisting}
            sum' [a,b,c] = a + b + c
        \end{lstlisting}
    \end{eg}

    \begin{proof}
        Vi har at

        \begin{lstlisting}
            sum' [a,b,c]
            = sum (a:b:c:[])
            = a + sum'(b:c:[])
            = a + b + sum'(c:[])
            = a + b + c + sum' []
            = a + b + c + 0
            = a + b + c
        \end{lstlisting}
    \end{proof}

    \subsection{Forskjellige typer rekursjon}
    \begin{itemize}
        \item Rekursjon på heltall
        \item Rekursjon på lister (og senere datatyper)
        \item Produktiv rekursjon
        \item Generell rekursjon
    \end{itemize}

    \subsubsection{Rekursjon på heltall}

    \begin{lstlisting}
        countDown :: Integer -> [Integer]
        countDown 0 = [0]
        countDown x = x:(counDown (x - 1))
    \end{lstlisting}

     \begin{itemize}
         \item Oftests er 0 eller 1 grunntilfellet.
         \item De rekursive kallene kaller funksjonen med et midnre parameter.
     \end{itemize}

     OBS: Dette gir ofte partielle funksjoner.

     \begin{itemize}
         \item Hva skjer med funksjonen over hvis man gir negativ input?
             \begin{itemize}
                 \item Jo den vil kalle på seg selv i all evighet ettersom den aldri når grunntilfellet.
             \end{itemize}
         \item Hvordan burde vi ordne det slik at vi ikke får problemer med negative tall?
             \begin{itemize}
                 \item Enten ved å sjekke om tallet er negativt, og dersom det er det, legger vi til 1 istedenfor og trekke fra, eller så kan vi kalle funksjonen på absoluttverdien av \( x \).
             \end{itemize}
     \end{itemize}

     \subsection{Primitiv vs generell rekursjon (Ackermann)}

     Men det finnnes veldig kompliserte ting man kan gj;re hvis man vil:

     \begin{lstlisting}
         ackermann :: Integer -> Integer -> Integer
         ackermann 0 n = n + 1
         ackermann m 0 = ackermann (m - 1) 1
         ackermann m n = ackermann (m - 1) (ackermann m (n - 1))
     \end{lstlisting}

     Denne funksjonen vokser veldig fort! Men i teorien kommer den alltid frem til et svar.

     \subsection{Rekursjon på lister og andre datastrukturer}

     Vi har allerede sett et eksempel, la oss ta noen til

     \begin{eg}
         For å reversere en liste rekursivt

         \begin{lstlisting}
             reverse :: [a] -> [a]
             reverse [] = []
             reverse (a : as) = reverse as ++ [a]
         \end{lstlisting}

         \begin{itemize}
             \item Gjør mønster matching på alle konstruktørene.
             \item Det rekursive kallet gjøres direkte på variablene introdusert av mønsteret.
         \end{itemize}
     \end{eg}

     \begin{eg}
         For å definere zip

         \begin{lstlisting}
             
         \end{lstlisting}
     \end{eg}
    
\end{document}
