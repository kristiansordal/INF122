\documentclass{article}
\input{preamble.tex}

\begin{document}
    \section{Forelesning 7}

    \subsection{Dypere mønstermatching}

    \begin{eg}
        Skriv en funksjon som sjekker om en typer er sortert.

        \begin{enumerate}
            \item Hva er typen til en slik funksjon?
            \item Hva må funksjonen gjøre for å sjekke at en liste er sortert?
            \item Hvis vi vil bruke rekursjon, hvilke mønster kan vi bruke?
        \end{enumerate}

        \begin{lstlisting}
            sorted :: (ord a) => [a] -> Bool
        \end{lstlisting}
    \end{eg}

    \begin{eg}
        Vi kan også matche hvert enkelt element i listen, basert på typen

        \begin{lstlisting}
            filterEmpty:: [[a]] -> [[a]]
            filterEmpty [] = []
            filterEmpty ([]:xs) = xs
            filterEmpty ([]:xs) = xs
            filterEmpty (x:xs)
        \end{lstlisting}
    \end{eg}

    \subsection{Gjensidig rekursjon}
    To funksjoner kan v1're definert i termer av hverandre:

    \begin{lstlisting}
        odds :: [a] -> [a]
        odds [] = []
        odds (x:xs) = evens xs

        evens :: [a] -> [a]
        evens [] = []
        evens (x:xs) = x : odds xs
    \end{lstlisting}

    \subsection{Egendefinerte datatyper}
    I haskell kan man innføre nye datatyper ved hjelp av nøkkelordet data:

    \begin{lstlisting}
        data CelestialObject = Star String Integer
                             | Planet String String 
                             | Moon String String 
    \end{lstlisting}

    Deklarasjonen består av

    \begin{itemize}
        \item Navn på datatypen: \texttt{CelestialObject}
        \item Liste med konstruktører: \texttt{Star, Planet, Moon}
        \item Argumenter til konstruktøren (data som lagres i elementene)
    \end{itemize}

    Når en datatyper er definert kan vi lage elementer i den:

    \begin{lstlisting}
        solarSystem :: [CelestialObject]
        solarSystem = [Star "The Sun" 4600000000,
                       Planet "Mercury" "The Sun",
                       Planet "Venus" "The Sun",
                       Planet "Earth" "The Sun", Moon "The Moon" "Earth"
                       Planet "Mars" "The Sun", Moon "Phobos" "Mars"
                                              , Moon "Deimos" "Mars"]

    \end{lstlisting}

    Vi kan også definere funksjoner ved hjelp av mønster

    \begin{lstlisting}
        displayInfo :: CelestialObject -> String
        displayInfo (Star name age)
        = "The star " ++ name ++ " is " ++ age ++ " years old."
    \end{lstlisting}
\end{document}
