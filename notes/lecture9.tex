\documentclass{article}
\input{preamble.tex}

\begin{document}
    \section{Forelesning 9}

    \subsection{Typeklasser}
    Typeklassene ligner på "interface" i Java. Typeklasser er mer fleksible:

    \begin{itemize}
        \item Et Java interface må defineres før klassen.
        \item En typeklasse kan defineres når som helst.
    \end{itemize}

    \begin{idea}
        Noen funknsjoner fungerer for alle typer (polymorfe), men noen fungerer bare på typer med en viss struktur. For eksempel likhet, rodning, eller \( + \) og \( - \).
    \end{idea}

    En typeklasse bestemmer typen til en liste med funksjoner. En hver type som kører til typeklassen skriver in egen implementasjon.

    \begin{eg}
        Klassen \texttt{Eq} er den enkleste innebygde klassen:

        \begin{lstlisting}
        class Eq aa where
            (==) :: a -> a -> Bool 
        \end{lstlisting}

    For eksempel Bool:

    \begin{lstlisting}
    data Bool = True | False

    instance Eq Bool where
        instance Eq Bool where        
        True == True = True
        False == False = True
        _ == _ = False
    \end{lstlisting}    

    For eksempel for lister:

    \begin{lstlisting}
    instance (Eq a) => Eq [a] where
    [] == [] = True
    (a:as) == (b:bs) = (a == b) && (as == bs)
    _ == _ = False
    \end{lstlisting}
    \end{eg}

    \begin{eg}
        Eksempel på rotering av lister
        \begin{lstlisting}
        class Rotateable a where 
            rotateLeft :: a -> a
            rotateRight :: a -> a

        instance Rotateable [b] where
            rotateLeft [] = []
            rotateLeft (a:as) = as ++ [a]
            rotateLeft [] = []
            rotateLeft as = last as : init as 
        \end{lstlisting}
    \end{eg}

    \subsection{Binære søkretrær}

    \subsubsection{Antagelse}
    Vi antar at vi jobber med en type som implementerer Ord typen. (F. eks heltall, eller strenger)
    \subsubsection{Basis ide}
    Binære søketrær er basert på samme ide som QuickSort: Hvis vi har et element \( x \) kan vi dele opp en hver mengde i tre deler:

    \begin{itemize}
        \item De som er mindre enn \( x \) (til venstre)
        \item Elementet \( x \) selv (i midten)
        \item De som er større enn \( x \) (til høyre)
    \end{itemize}



\end{document}
