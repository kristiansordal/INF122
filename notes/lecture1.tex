\documentclass[twocolumn]{article}
\input{preamble.tex}

\begin{document}
\section{Forelesning 1}
\subsection{Plan for forelesningen}
\begin{itemize}
    \item Gjennomgang av forventinger
    \item Online ressurser
    \item Eksempler på programmer i Haskell
    \item Funksjoner i Haskell
    \item Strukturen til et Haskellprogram
\end{itemize}

\subsection{Online Haskellressurser}

\begin{itemize}
    \item \textit{Learn You a Haskell}: learnyouahaskell.com
    \item \textit{Haskell wikibok}: en.m.wikibooks.orgs/wiki/haskell
    \item \textit{Hoogle}: hoogle.haskell.ord
    \item \textit{Mer}: haskell.ord/documentation
\end{itemize}

\subsection{Eksempler på programmer laget i Haskell}
\begin{itemize}
    \item Pandoc
    \item Xmonad
    \item Darcs
    \item GF - Grammatical Framework
    \item GitHub's semantic tool
\end{itemize}

Og andre diverse selskaper som Standard Chartered og Klarna.

\subsection{Funksjoner}
Hva er en funksjon? Vi bruker en funksjon ved å få en verdi ved å gi den et argument.
\bigskip

I matematikken brukes \( f\left( x \right) \) for å bruke en funksjon \( f \) på en verdi \( x \). Hvis funksjonen tar imot flere argumenter skriver man \( f\left(x,y,z\right) \) for å gi dem.
\bigskip

I Haskell droppes parantesene, og man skriver bare \verb!f x!, og dersom det er flere argumenter skrives det \verb!f x y z!. For å sette sammen funksjoner, må vi likevel bruke paranteser: \verb!f (g x)!. Dersom vi hadde skrevet \verb!f g x! ville vi gitt to argumenter til funksjonen.

\subsection{Haskellprogrammer}
Filnavn i haskell slutter på \verb!.hs! - ellers er hver fil ofte en \textit{modul}, hvor filnavn ofte er det samme som modulnavn. Modulnavn kommer øverst i filen, og er på formen \verb!module moduleName where!. Verdien \verb!main! er en spesiell verdi som har typen \verb!IO ()!. For å lage en kjørbar fil må main verdien ligge i modulen main.

\subsubsection{Presidensregler}
I Haskell binder funksjonene sterkes, det vil si at koden under tolkes på følgende måte.
% \begin{spverbatim}
%     f x = x + 1 .1
%     f 1 * 3 = (f 1) * 3 = 12 .2
% \end{spverbatim}

% Hvis det var meningen at det skulle tolkes på annen måte, må dette eksplisitt skrives

% \begin{spverbatim}
%     f (1 * 3).1
% \end{spverbatim}

% asdf

\end{document}
