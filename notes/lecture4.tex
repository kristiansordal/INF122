\documentclass{article}
\input{preamble.tex}

\begin{document}
    \section{Forelesning 4}

    \subsection{Plan for forelesning}

    Vi fortsetter fra forrige gang:
    \begin{itemize}
        \item Eksempler på listefunksjoner
        \item Assosiasjonslister
        \item Map
        \item Listekomprehensjon
        \item (Bonus: Isomofi av lister)
    \end{itemize}

    \subsection{Rush}

    Med unntak av 1979, utga rockebandet "Rush" ett album hvert å i perioden 1976-1982.

    \subsection{Nyttige listefunksjoner}


    \begin{align*}
        &\texttt{elem :: (Eq  a) => a -> [a] -> Bool} \\
        &\texttt{words :: String -> [String]} \\
        &\texttt{unwords :: [String] -> String} \\
    \end{align*}

    \begin{eg}
        Skriv en funksjon som fjerner alle ord som inneholder bokstaven "e".

    \end{eg}

    \begin{align*}
        &\texttt{removee :: String -> String} \\
        &\texttt{removee text = unwords (filter (notElem 'e') (words text))} \\
    \end{align*}

    \subsection{Zip}

    Zip er en funksjon med typing

    \[ \texttt{zip :: [a] -> [b] -> [(a,b)]} \]

    \subsection{Currying: Forklaring}

    En funksjon som tar to argumenter har vi sett kan ha type på formen:

    \[ \texttt{A -> B -> C} \]

    Dette kalles "currying" 

    \subsection{Lookup}

    Den nyttigste funksjonen for assisiasjonslister:

    \[ \texttt{lookup :: a -> [(a,b)] -> Maybe b} \]

    \subsection{Isomorfi av typer: Definisjon}
    Vi sier at to typer \( A \) og \( B \) er isomorfe dersom vi kan funne funksjoner

    \begin{align*}
        &\texttt{f :: A -> B og}\\
        &\texttt{g :: B -> A}
    \end{align*}
\begin{figure}[H]
	\centering
	\incfig{inf122test}
\end{figure}

\end{document}
