\documentclass{article}
\input{preamble.tex}

\begin{document}
    \section{Forelesning 3}

    \subsubsection{Plan for forelesningen}

    Vi begynner på lista fra forrige gang:

    \begin{itemize}
        \item Tuppler
        \item Maybe
        \item Lister
        \item Either
        \item Map
    \end{itemize}

    Vi kommer definitivt ikke lenger enn til lister idag.

    \subsection{Tuppler: Eksempeltyper}

    Tuppeltyper skrives med paranteser og komma:

    \begin{itemize}
        \item \texttt{(Integer, String)} er typen av alle par av heltall og strenger.
        \item \texttt{(Double,Double,Double)} er typen av alle 3D koordinater, eller vektorer, med dobbel presisjonsflyttall.
    \end{itemize}
    \begin{align*}
        & \vdots \\
        &\text{osv}
    \end{align*}

    \subsubsection{Sammenligning med lister}

    \begin{itemize}
        \item Tuppler: Fiksert antall posisjoner for data.
        \item Hver posisjon har en type.
        \item Lister: Variabelt antall posisjoner for data (lengde)
        \item Lister: Alle elementer i listen har samme type.
    \end{itemize}

\subsubsection{Kanonsike verdier}

De kanoniske verdiene i tuppeltypene er de på formen:

\subsubsection{Funksjoner definert med møn}

\begin{eg}
    La oss skrive en funksjon som normaliserer en vektor. (Det vil si å skalere den slik at den har lengde 1.)

    \begin{figure}[H]
        \begin{center}
            \includegraphics[width=0.95\textwidth]{normvec.png}
        \end{center}
    \end{figure}
\end{eg}

Generelt definerer vi mønster matching hvordan funksjonen opererer på en kanonisk verdi av typen.
\medskip

\( \left( 1,2,3 \right) \) er en kanonisk verdi av typen \texttt{(Double, Double, Double)}

\subsubsection{Projeksjoner}
For par (tupler med to posisjoner), har vi to funksjoner for å hente ut data fra et par:

\begin{align*}
    &\texttt{fst :: (a, b) -> a} \\
\end{align*}

\subsection{Currying}
En funksjon som tar to verdier har vi sett kan ha type på formen:

\[ \texttt{A -> B -> C} \]

Eksempel fra boken

\begin{align*}
    &\texttt{add :: (Integer, Integer) -> Integer} \\ 
    &\texttt{add (x, y) = x + y}\\
    &\texttt{add' :: Integer -> Integer -> Integer}\\
    &\texttt{add' x y = x + y}
\end{align*}

\subsubsection{Fordeler med begge}
Fordelen med currying er partiell applikasjon

\[ \texttt{add' 3 :: Integer -> Integer} \]

Fordelen med uncurrying er at det er lett å mappe funksjonen inn i en struktur:

\[ > map add 3 [(1,2), (2,1), (0,3)] \]

\subsubsection{Funksjoner for å gå frem og tilbake}

Gjøre en funksjon curried:

\[ \texttt{curry :: ((a,b) -> c) -> a -> b -> c} \]

Gjøre en funksjon uncurried:

\[ \texttt{uncurry :: (a -> b -> c) -> (a,b) -> c} \]

\subsubsection{Hva er konvensjonen i Haskell}

Standard i Haskell er å definere funksjonen som ferdig curry-et, og heller bruke uncurry eksplisitt der det er behov.

\[ \texttt{map (uncurry (+)) [(x,y) | x <- [0..3], y <- [x..3]]} \]

\subsection{Maybe}

Vi lager \textit{Maybe-typer} ved å skrive \texttt{Maybe} foran typen

\begin{itemize}
    \item \texttt{Maybe Integer} er typen av kanskje heltall
\end{itemize}

\begin{eg}
    Eksempler på elementer

    \begin{align*}
        &\texttt{Just 3 :: Maybe Integer} \\
        &\texttt{Nothing :: Maybe Integer}
    \end{align*}

    \[ \texttt{readMaybe :: String -> Maybe Integer} \]

    \begin{note}
        \texttt{readMaybe} må importeres fra \texttt{Text.Read}
    \end{note}
\end{eg}

\subsubsection{Konstruktører}

Maybe typene har to konstruktører
\begin{itemize}
    \item \texttt{Nothing :: Maybe a}
    \item \texttt{Just :: a -> Maybe a}
\end{itemize}

\subsubsection{Nyttige funksjoner}

La oss se på noen nyttige funksjoner

\begin{itemize}
    \item \texttt{maybe :: b -> (a -> b) -> Maybe a -> b}
    \item \texttt{fromMaybe :: a -> Maybe a -> a}
    \item \texttt{fmap :: (a -> b) -> Maybe a -> Maybe b}
\end{itemize}

Man kan også få ut boolske verdier fra en Maybe-verdi

\subsection{Lister}

\begin{itemize}
    \item \texttt{[Integer]} er en liste av heltall
    \item \texttt{[Char] = String} er en streng / liste med characters 
    \item \texttt{[[Integer]]} er en liste med lister av heltall
    \item \texttt{[Double -> Double]} er en liste med funksjoner på flyttal
\end{itemize}

\subsubsection{Konstruktører}

Lister har to konstruktører

\begin{itemize}
    \item \texttt{[] :: [a]} - Den tomme listen
    \item \texttt{(:) :: a -> [a] -> [a]} - Legger et element foran i listen
\end{itemize}

Dermed er kanoniske element i en listetype \texttt{[A]} de som er på formen

\begin{itemize}
    \item \texttt{[]} eller
    \item \texttt{a : as} hvor \texttt{a :: A} og \texttt{as :: [A]}
\end{itemize}

Syntaktisk sukker \texttt{[1,2,3,]} istedetfor \texttt{1:2:3:[]}

\subsubsection{Mønster}

Vi kan definere funksjoner på lister ved hjelp av mønster som matcher \texttt{(:)} og \texttt{[]:}

\begin{align*}
    &\texttt{safeHead :: [a] -> Maybe a} \\
    &\texttt{safeHead}
\end{align*}

\subsubsection{Rekursjon}

Senere skal vi se hvordan vi kan definere alle mulige funksjoner på lister vha. rekusjon

\begin{align*}
    &\texttt{duplicate :: [a] -> [a]} \\
    &\texttt{dublicate [] = []} \\
    &\texttt{duplicate (a : as) = a : a : duplicate as}
\end{align*}
\end{document}
