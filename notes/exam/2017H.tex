\documentclass{article}
\input{preamble.tex}

\begin{document}
    \section{Programmer Følgende Funksjoner}

    \textbf{1.1} \texttt{harEl:: (t -> Bool) -> [t] -> Bool}, slik at \texttt{harEl pr xs = True} hvis listen \texttt{xs} har et element \texttt{x} som tilfredsstiller predikatet \texttt{pr}, dvs, \texttt{pr x = True} og \texttt{False} ellers. F.eks:
    \begin{lstlisting}
harEl (==3) [1,1,2,3,2] = True
harEl (<3) [1,1,2,3,2] = True
harEl (>5) [1,1,2,3,2] = False
    \end{lstlisting}

    \begin{ans}
        Bruker \texttt{any}.
        \begin{lstlisting}
harEl :: (t -> Bool) -> [t] -> Bool
harEl = any
        \end{lstlisting}
    \end{ans}

    \textbf{1.1} \texttt{el :: (t -> Bool) -> [t] -> t} slik at \texttt{ el pr xs} returnerer det første elementet \texttt{x} fra listen \texttt{xs} som tilfredsstiller predikatet \texttt{pr}. Funksjonen antar at et slikt element finnes i listen. F.eks:

    \begin{lstlisting}
el ((=='a').fst) [('b',2), ('a',3),('a',4)] = ('a',3)
el ((>3).snd) [('b',2), ('a',3), ('a',4)] = ('a',4)
    \end{lstlisting}

    \begin{ans}
        Bruker \texttt{head} og \texttt{filter}.
        \begin{lstlisting}
el :: (t -> Bool) -> [t] -> t
el pr = head . filter pr
        \end{lstlisting}
    \end{ans}

    \textbf{1.3} \texttt{gRep :: (t -> Bool) -> t -> [t] -> [t]} slik at \texttt{gRep pr y xs} erstatter med \texttt{y}, ethvert element \texttt{x} fra listen \texttt{xs} som tilfredsstiller predikatet \texttt{pr}. F. eks:

    \begin{lstlisting}
gRep (<'d') 'z' "abcd" = "zzzd" 
gRep (=='a') 'x' "abcbcac" = "xbcbcxc" 
    \end{lstlisting}

    \begin{ans}
        Bruker \texttt{map} og en lambda funksjon
        
        \begin{lstlisting}
gRep :: (t -> Bool) -> t -> [t] -> [t]
gRep pr y = map (\x ->
  case () of
  _ | pr x -> y
    | otherwise -> x
  )
        \end{lstlisting}
    \end{ans}

\textbf{1.4} Vi bruker Binære trær med heltall lagret i alle noder (inklusivt blader) definert ved \texttt{data BT = B Int | N BT Int BT}. Programmer følgende funksjoner:

\begin{enumerate}[label=\alph*)]
    \item \texttt{elf :: BT -> Int -> Bool}, slik at \texttt{elt tr x = True} hvis tallet \texttt{x} forekommer i treet \texttt{tr}, og \texttt{False} ellers. F.eks \texttt{elt (N (B 1) 3 (B 0)) 2 = True} of \texttt{elt (B 1) 2 = False}.
    \item \texttt{toL :: BT -> [Int]} slik at \texttt{toL tr} er en liste med alle tall som forekommer i i treet \texttt{tr}.
    \item \texttt{dup :: BT -> Bool} slik at \texttt{dup tr = True} hvis noen tall forekommer (minst) to ganger i treet \texttt{tr} og \texttt{False} ellers.
\end{enumerate}

\begin{ans}
    \textbf{a)} Bruker rekursjon, pattern matching og guards
    \begin{lstlisting}
elt :: BT -> Int -> Bool
elt (B val) x = val == x
elt (N left val right ) x
  | val == x = True
  | val /= x = elt left x || elt right x
\end{lstlisting}
\bigskip

    \textbf{b)} Bruker pattern matching
    \begin{lstlisting}
toL :: BT -> [Int]
toL (B x) = [x]
toL (N left val right) = toL left ++ [val] ++ toL right
    \end{lstlisting}

    \textbf{c)} Bruker en hjelpefunksjon
    \begin{lstlisting}
dup :: BT -> Bool
dup = dupL . toL

dupL :: [Int] -> Bool
dupL [] = False
dupL (x:xs) = elem x xs || dupL xs
    \end{lstlisting}
\end{ans}

\section{Rettede Grafer}
\textbf{2.1} Programmer en funksjon \texttt{naboL :: Eq t => [(t,t)]->[(t,[t])] s} som konverterer en kantliste representasjon til en naboliste


\end{document}
