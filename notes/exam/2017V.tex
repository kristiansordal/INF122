\documentclass{article}
\input{preamble.tex}

\begin{document}
    \section{Problem 2 - Higher Order Functions}

    (a) Given the function \texttt{foldr} as below

    \begin{lstlisting}
foldr :: (a -> b -> b) -> b -> [a] -> b 
foldr f v [] = v
foldr f v (x:xs) = f x (foldr f v xs)
    \end{lstlisting}

    Using the function \texttt{foldr}, define a function

    \[ \texttt{lengthsum :: (Num a, Num b) => [a] -> (b, a)} \]

    That takes a list of numbers as input, then return the length and the sum of the list as a pair.

    \begin{ans}
    \hspace{1em}
     \begin{lstlisting}
lengthSum :: (Num a, Num b) => [a] -> (b, a)
lengthSum = foldr (\n (x, y) -> (1 + x, n + y)) (0, 0)
     \end{lstlisting}  
    \end{ans}

    (b) Given the function \texttt{foldl} as below:
            
    \begin{lstlisting}
foldl :: (a -> b -> a) -> a -> [b] -> a 
foldl f v [] = v
foldl f v (x:xs) = foldl f (f v x) xs
    \end{lstlisting}

    Using the function \texttt{foldl} define a function

    \[ \texttt{inList :: (Eq a) => a -> [a] -> Bool} \]

    that takes a value and a list of the same type as inputs, then checks wheter the value is an element of the list.

    \begin{ans}
        \hspace{1em}
        \begin{lstlisting}
inList :: (Eq a) => a -> [a] -> Bool
inList x = foldl (\acc y -> (x == y) || acc) False
        \end{lstlisting}
    \end{ans}


    (c) What do the following expressions return?
    \medskip

    (i) \texttt{foldr (:) "hello" "world!"} = "world!hello"
    \medskip

    (ii) 
    \begin{lstlisting}
       foldl (\xs -> \x -> x:xs) "INF122" "exam" -> 
    \end{lstlisting}

    \section{Problem 3 - An evaluator}
    Consider the following type declaration:

    \[ \texttt{data Expr = V Int | M Expr Expr | D Expr Expr} \]

    You are asked to implement an evaluator

    \[ \texttt{eval :: Expr -> Maybe Int} \]

    which evaluates an expression of type Expr defined above

    \begin{ans}
        \hspace{1em}

        \begin{lstlisting}
data Expr = V Int | M Expr Expr | D Expr Expr

div' :: Int -> Int -> Maybe Int
div' _ 0 = Nothing
div' x y = Just (x `div` y)

eval :: Expr -> Maybe Int
eval (V x) = Just x
eval (M r l) =
  case eval r of
    Nothing -> Nothing
    Just x -> case eval l of
      Nothing -> Nothing
      Just y ->
        if x >= 0 && y >= 0
          then Just (x * y)
          else Nothing
eval (D r l) =
  case eval r of
    Nothing -> Nothing
    Just x -> case eval l of
      Nothing -> Nothing
      Just y -> div' x y
        \end{lstlisting}
    \end{ans}

    \section{Problem 5 - Input and Output}

    In this problem, you are not supposed to use any build-in functions in Haskell except \texttt{return} and list operators
    \medskip

    (a) Given a lsit of \texttt{IO}-actions, implement a function

    \[ \texttt{toDoList :: [IO a] -> IO [a]} \]

    which excecutes each of the \texttt{IO}-actions in the list and gives back an \texttt{IO}-action that returns a list containing the corresponding results in the same order as output.

    \begin{ans}
        \hspace{1em}
        \begin{lstlisting}
toDoList :: [IO a] -> IO [a]
toDoList [] = return []
toDoList (x : xs) = do
  action <- x
  list' <- toDoList xs
  return (action : list')
        \end{lstlisting}
    \end{ans}

    (b) Implement a map function for \texttt{IO}-actions

    \[ \texttt{mapActions :: (a -> IO b) -> [a] -> IO [b]} \]

    which takes a functions of type \texttt{a -> IO b}, and then applies this function to each item in an input list of type \texttt{[a]}. The \texttt{mapActions} function finally gives bac an \texttt{IO}-action that returns a list.

    \begin{ans}
        \hspace{1em}
        \begin{lstlisting}
mapActions :: (a -> IO b) -> [a] -> IO [b]
mapActions f [] = return []
mapActions f (x : xs) = do
  y <- f x
  ys <- mapActions f xs
  return (y : ys)
        \end{lstlisting}
    \end{ans}


\end{document}
