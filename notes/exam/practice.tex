\documentclass{article}
% Some basic packagesLLuu
\usepackage[utf8]{inputenc}
\usepackage{spverbatim}
\usepackage[margin=1.2in]{geometry}
\usepackage{textcomp}
\usepackage{url}
\usepackage{graphicx}
\usepackage{float}
\usepackage{enumitem}
\usepackage{standalone}
\usepackage{tcolorbox}
\usepackage{wrapfig}
% \usepackage{svg}
% \usepackage{svg-inkscape} 

\graphicspath{{./figures}}

%color settings
\usepackage{xcolor}
\definecolor{gruvbgdark}{HTML}{1d2021}
\definecolor{gruvtextdark}{HTML}{ebdbb2}
\definecolor{gruvbglight}{HTML}{f9f5d7}
\definecolor{gruvtextlight}{HTML}{3c3836}
\definecolor{NavyBlue}{HTML}{266bbd}
\definecolor{RawSienna}{HTML}{94330e}
\definecolor{ForestGreen}{HTML}{149b52}
% \pagecolor{gruvbgdark}
% \color{gruvtextdark}

% Hide page number when page is empty
\usepackage{emptypage}
\usepackage{subcaption}
\usepackage{multicol}

% Math stuff
\usepackage{amsmath, amsfonts, mathtools, amsthm, amssymb}
% Fancy script capitals
\usepackage{mathrsfs}
\usepackage{cancel}

% Bold math
\usepackage{bm}

% SVG setup
% \svgsetup{inkscapeexe=inkscape, inkscapearea=drawing}
% \svgpath{~/dev/DAVE3700-Matte-3000/figures/}

% Some shortcuts
\newcommand\N{\ensuremath{\mathbb{N}}}
\newcommand\R{\ensuremath{\mathbb{R}}}
\newcommand\Z{\ensuremath{\mathbb{Z}}}
\renewcommand\O{\ensuremath{\emptyset}}
\newcommand\Q{\ensuremath{\mathbb{Q}}}
\newcommand\C{\ensuremath{\mathbb{C}}}

%Make implies and impliedby shorter
\let\implies\Rightarrow
\let\impliedby\Leftarrow
\let\iff\Leftrightarrow

\let\epsilon\varepsilon

% Add \contra symbol to denote contradiction
% \usepackage{stmaryrd} % for \lightning
% \newcommand\contra{\scalebox{1.5}{$\lightning$}}

\let\phi\varphi

% Command for short corrections
% Usage: 1+1=\correct{3}{2}

\definecolor{correct}{HTML}{009900}
\newcommand\correct[2]{\ensuremath{\:}{\color{red}{#1}}\ensuremath{\to }{\color{correct}{#2}}\ensuremath{\:}}
\newcommand\green[1]{{\color{correct}{#1}}}

% horizontal rule
% \newcommand\hr{
%     \noindent\rule[0.5ex]{\linewidth}{0.5pt}
% }

% hide parts
\newcommand\hide[1]{}

% Environments
\makeatother

% For box around Definition, Theorem, \ldots
% theorems
\usepackage{thmtools}
\usepackage[framemethod=TikZ]{mdframed}
\mdfsetup{skipabove=1em,skipbelow=1em, innertopmargin=5pt, innerbottommargin=6pt}

\theoremstyle{definition}

\makeatletter

% \declaretheoremstyle[headfont=\bfseries, bodyfont=\normalfont, mdframed={ nobreak } ]{thmgreenbox}
% \declaretheoremstyle[headfont=\bfseries, bodyfont=\normalfont, mdframed={ nobreak } ]{thmredbox}
% \declaretheoremstyle[headfont=\bfseries, bodyfont=\normalfont, spaceabove=0.5cm, spacebelow=0.5cm]{thmbluebox}
% % \declaretheoremstyle[headfont=\bfseries, bodyfont=\normalfont]{thmbluebox}
% \declaretheoremstyle[headfont=\bfseries, bodyfont=\normalfont]{thmblueline}
% \declaretheoremstyle[headfont=\bfseries, bodyfont=\normalfont, numbered=no, mdframed={ rightline=false, topline=false, bottomline=false, }, qed=\qedsymbol ]{thmproofbox}
% \declaretheoremstyle[headfont=\bfseries\sffamily, bodyfont=\normalfont, numbered=no, mdframed={ nobreak, rightline=false, topline=false, bottomline=false } ]{thmexplanationbox}
\declaretheoremstyle[headfont=\bfseries, bodyfont=\normalfont, numbered=no]{idea}

\declaretheoremstyle[
	headfont=\bfseries\color{ForestGreen!70!black}, bodyfont=\normalfont,
	mdframed={
			linewidth=2pt,
			rightline=false, topline=false, bottomline=false,
			linecolor=ForestGreen, backgroundcolor=ForestGreen!5,
		}
]{thmgreenbox}

\declaretheoremstyle[
	headfont=\bfseries\color{NavyBlue!70!black}, bodyfont=\normalfont,
	mdframed={
			linewidth=2pt,
			rightline=false, topline=false, bottomline=false,
			linecolor=NavyBlue, backgroundcolor=NavyBlue!5,
		}
]{thmbluebox}

\declaretheoremstyle[
	headfont=\bfseries\color{NavyBlue!70!black}, bodyfont=\normalfont,
	mdframed={
			linewidth=2pt,
			rightline=false, topline=false, bottomline=false,
			linecolor=NavyBlue
		}
]{thmblueline}

\declaretheoremstyle[
	headfont=\bfseries\color{RawSienna!70!black}, bodyfont=\normalfont,
	mdframed={
			linewidth=2pt,
			rightline=false, topline=false, bottomline=false,
			linecolor=RawSienna, backgroundcolor=RawSienna!5,
		}
]{thmredbox}

\declaretheoremstyle[
	headfont=\bfseries\color{RawSienna!70!black}, bodyfont=\normalfont,
	numbered=no,
	mdframed={
			linewidth=2pt,
			rightline=false, topline=false, bottomline=false,
			linecolor=RawSienna, backgroundcolor=RawSienna!5,
		},
	qed=\qedsymbol
]{thmproofbox}

\declaretheoremstyle[
	headfont=\bfseries\color{NavyBlue!70!black}, bodyfont=\normalfont,
	numbered=no,
	mdframed={
			linewidth=2pt,
			rightline=false, topline=false, bottomline=false,
			linecolor=NavyBlue, backgroundcolor=NavyBlue!1,
		},
]{thmexplanationbox}

\declaretheorem[style=thmgreenbox, name=Definisjon]{definition}
\declaretheorem[sibling=definition, style=thmredbox, name=Corollary]{corollary}
\declaretheorem[style=thmbluebox, numbered=no, name=Idea]{idea}
\declaretheorem[style=idea, style=thmredbox, name=Proposition]{prop}
\declaretheorem[sibling=definition, style=thmredbox, name=Theorem]{theorem}
\declaretheorem[sibling=definition, style=thmredbox, name=Lemma]{lemma}



\declaretheorem[numbered=no, style=thmexplanationbox, name=Proof]{explanation}
\declaretheorem[numbered=no, style=thmproofbox, name=Bevis]{replacementproof}
\declaretheorem[style=thmgreenbox,  numbered=no, name=Oppgave]{ex}
\declaretheorem[style=thmbluebox,  numbered=no, name=Eksempel]{eg} \declaretheorem[style=thmblueline, numbered=no, name=Remark]{remark}
\declaretheorem[style=thmblueline, numbered=no, name=Merk]{note}

\renewenvironment{proof}[1][\proofname]{\begin{replacementproof}}{\end{replacementproof}}

\AtEndEnvironment{eg}{\null\hfill$\diamond$}%

\newtheorem*{uovt}{UOVT}
\newtheorem*{notation}{Notation}
\newtheorem*{previouslyseen}{As previously seen}
\newtheorem*{problem}{Problem}
\newtheorem*{observe}{Observe}
\newtheorem*{property}{Property}
\newtheorem*{intuition}{Intuition}


% Exercise 
% Usage:
% \oefening{5}
% \suboefening{1}
% \suboefening{2}
% \suboefening{3}
% gives
% Oefening 5
%   Oefening 5.1
%   Oefening 5.2
%   Oefening 5.3
\newcommand{\oefening}[1]{%
	\def\@oefening{#1}%
	\subsection*{Oefening #1}
}

\newcommand{\suboefening}[1]{%
	\subsubsection*{Oefening \@oefening.#1}
}


% \lecture starts a new lecture (les in dutch)
%
% Usage:
% \lecture{1}{di 12 feb 2019 16:00}{Inleiding}
%
% This adds a section heading with the number / title of the lecture and a
% margin paragraph with the date.

% I use \dateparts here to hide the year (2019). This way, I can easily parse
% the date of each lecture unambiguously while still having a human-friendly
% short format printed to the pdf.

% \usepackage{xifthen}
% \def\testdateparts#1{\dateparts#1\relax}
% \def\dateparts#1 #2 #3 #4 #5\relax{
% 	\marginpar{\small\textsf{\mbox{#1 #2 #3 #5}}}
% }

% \def\@lecture{}%
% \newcommand{\lecture}[3]{
% 	\ifthenelse{\isempty{#3}}{%
% 		\def\@lecture{Lecture #1}%
% 	}{%
% 		\def\@lecture{Lecture #1: #3}%
% 	}%
% 	\subsection*{\@lecture}
% 	% \marginpar{\small\textsf{\mbox{#2}}}
% }

\usepackage{listings}

\definecolor{dkgreen}{rgb}{0,0.6,0}
\definecolor{gray}{rgb}{0.5,0.5,0.5}
\definecolor{mauve}{rgb}{0.58,0,0.82}

\lstset{frame=none,
  language=Haskell,
  aboveskip=3mm,
  belowskip=3mm,
  showstringspaces=false,
  columns=flexible,
  basicstyle={\small\ttfamily},
  numbers=none,
  numberstyle=\tiny\color{gray},
  keywordstyle=\color{blue},
  commentstyle=\color{dkgreen},
  stringstyle=\color{mauve},
  breaklines=true,
  breakatwhitespace=true,
  tabsize=3
}



% These are the fancy headers
\usepackage{fancyhdr}
\pagestyle{fancy}

% LE: left even
% RO: right odd
% CE, CO: center even, center odd
% My name for when I print my lecture notes to use for an open book exam.
\fancyhead[LE,RO]{Kristian Sørdal}

\fancyhead[RO,LE]{INF122 - Funksjonell Programmering} % Right odd,  Left even
% \fancyhead[RE,LO]{\leftmark}          % Right even, Left odd
\fancyhead[RE,LO]{Kristian Sørdal}          % Right even, Left odd

\fancyfoot[RO,LE]{\thepage}  % Right odd,  Left even
% \fancyfoot[RE,LO]{Kristian Sørdal}          % Right even, Left odd
\fancyfoot[C]{\leftmark}     % Center

\makeatother

% Todonotes and inline notes in fancy boxes
\usepackage{todonotes}
\usepackage{tcolorbox}

% Make boxes breakable
\tcbuselibrary{breakable}

% Figure support as explained in my blog post.
\usepackage{import}
\usepackage{xifthen}
\usepackage{pdfpages}
\usepackage{transparent}
\newcommand{\incfig}[2][1]{%
	% \begin{center}
	\def\svgwidth{#1\columnwidth}
	\import{./figures/}{#2.pdf_tex}
	% \end{center}
}
% Fix some stuff
% %http://tex.stackexchange.com/questions/76273/multiple-pdfs-with-page-group-included-in-a-single-page-warning
\pdfsuppresswarningpagegroup=1
\author{Kristian Sørdal}


\begin{document}
\section{Øvingsoppgaver}
    \subsection{Regne / Flervalgsoppgaver}

    \begin{enumerate}[label=\alph*)]
        \item Hva er verdien til uttrykket \texttt{map (+3) [1,2,3]}? \( \rightarrow  \) \texttt{[4,5,6]}
        \item Hva er verdien til uttrykket \texttt{sum [x + 3 | x <- [1,2,3]]}? \( \rightarrow \) \texttt{15}
        \item Hva er verdien til uttrykket \verb!(\!\texttt{x y -> x - y)} 7 3? \( \rightarrow \) \texttt{4}
        \item Hva er typen til uttrykket \verb!(\!\texttt{x -> 3 : tail x)}? \( \rightarrow \) \texttt{Num a => [Int] -> [Int]}
        \item Hva er riktig type til uttrykket \texttt{(3, Just "Haskell")}? \( \rightarrow \) \texttt{(Integer, Maybe String)}
        \item Hva er riktig type til uttrykket \texttt{Just (Left Nothing)}? \( \rightarrow \) \texttt{Maybe (Either (Maybe a) b)}
        \item Hvilken kind har \texttt{Either String}? \( \rightarrow \) \texttt{* -> * -> *}
        \item Hvilken Kind har \texttt{Integer -> Integer} \( \rightarrow \) \texttt{*}
        \item Hva er typen til funksjonen \texttt{f x y = if x then y else y * 3}?
            \begin{itemize}
                \item \texttt{f :: Bool -> Integer -> Integer}
                \item \texttt{f :: (Num a) -> Bool -> a -> a}
            \end{itemize}
    \end{enumerate}

    \subsection{Enkel IO}
    Skriv et program som leser inn et navn på formen "Fornavn Etternavn" og returnerer "Etternavn, Fornavn"

    \begin{ans}
        \hspace{1em}
        \begin{lstlisting}
    main = do
      name <- parse <$> getLine
      print name

    parse :: String -> String
    parse name = last (words name) ++ ", " ++ head (words name)
            
        \end{lstlisting}
    \end{ans}

\subsection{Listeoperasjoner}
Husk at \texttt{concat :: [[a]] ->[a]} kan brukes til å sette sammen en liste med strenger til en streng.

Konsonantene i språket vårt er \texttt{"bcdfghjklmnpqrstvwxz"}.

\begin{enumerate}[label=\alph*)]
    \item Skriv en funksjon \texttt{isConsonant :: Char -> Bool} som sjekker om en char er en konsonant på norsk.
        \begin{lstlisting}
isConsonant :: Char -> Bool
isConsonant c = c `elem` "bcdfghjklmnpqrstvwxz"
        \end{lstlisting}

Her er en funksjon som oversetter til røverspråket:

\begin{lstlisting}
translate :: String -> String
translate word = concat [if isConsonant x then [x] ++ "o" ++ [x] else [x] | x <- word]
\end{lstlisting}

\item Funksjonen translate bruker listekomprehensjon, skriv den slik at den bruker map istedet.
    \begin{lstlisting}
translate :: String -> String
translate = concatMap (\x -> if isConsonant x then [x] ++ "o" ++ [x] else [x])
    \end{lstlisting}

\item Skriv funksjonen \texttt{translate} slik at den bruker do-notasjon for lister

    \begin{lstlisting}
translate :: String -> IO String
translate (x : xs) = do
  if isConsonant x
    then return $ [x] ++ "o" ++ [x] ++ translate xs
    else return $ x : translate xs
    \end{lstlisting}

\item Skriv en funksjon \texttt{differences :: [Integer] -> [Integer]} som regner ut alle positive differanser mellom elementene i en liste, ved hjelp av listekomprehensjon. Eksempel: \texttt{differences [1,2,3] = [1,2,1]} fordi \( 2 - 1 = 1 \), \( 3 - 1 = 2 \) og \( 3 - 2 = 1 \).

\item Skriv en funksjon \texttt{everyOther :: [a] -> [a]}, som fjerner annethvert element fra en liste. Behold det første elementet i listen, fjern det andre, behold det tredje osv. Eksempel: \texttt{everyOther [1,2,3,4] = [1,3]}
    \begin{lstlisting}
everyOther :: [a] -> [a]
everyOther [] = []
everyOther [x] = [x]
everyOther (x : y : ys) = x : everyOther ys
        
    \end{lstlisting}
\end{enumerate}

    \subsection{Map}
    I denne oppgaven skal vi se på hvordan vi kan bruke maps til å representere en graf hvor vi har merket kantene.

    \[ \texttt{type Graph label node = Map node (Map label node)} \]

    Hver node mappes til et map som forteller hvilken node som ligger i enden til en kan med en viss merkelapp (\texttt{label}). Her er en graf med tre noder hvor merkelappene er bokstaver

    \begin{lstlisting}
data N = A | B | C 

graph0 :: Graph Char N
graph0 = Map.fromList [(A,Map.fromList [('r',B)])
                       ,(B,Map.fromList [('o',B),('t',C)])
                       ,(C,Map.fromList [('e',A),('t',C)])]
    \end{lstlisting}


    \begin{enumerate}[label=\alph*)]
        \item Skriv en funksjon som setter inn en kant med en gitt merkelapp mellom to noder i en graf.
            \begin{lstlisting}
insertLabeledEdge :: (Ord node) => Graph label node -> node -> node -> label -> Graph label node
insertLabeledEdge g n1 n2 l = Map.insert n1 (Map.singleton l n1) g
            \end{lstlisting}
        \item Bruk do-notasjon for Maybe til å skrive en funksjon som slår opp en node og en label i en graf og gir den neste noden

            \begin{lstlisting}
goNext :: (Ord node, Ord label) => Graph label node -> node -> label -> Maybe node
goNext graph start label = do
  labelMap <- Map.lookup start graph
  Map.lookup label labelMap
            \end{lstlisting}

        \item Skriv \texttt{goNext} ved hjelp av \texttt{>>=} operatoren istedet for do-notasjon.

            \begin{lstlisting}
goNext' :: (Ord node, Ord label) => Graph label node -> node -> label -> Maybe node
goNext' graph start label = Map.lookup start graph >>= Map.lookup label
            \end{lstlisting}
        \item Skriv en rekursiv funksjon som følger en liste med label fra en start node til en sluttnode

            \begin{lstlisting}
followPath :: (Ord node, Ord label, Ord N) => Graph label node -> node -> [label] -> Maybe node
followPath g n (x : xs) =
  case goNext' g n x of
    (Just node) -> followPath g node xs
    Nothing -> Just n
            \end{lstlisting}

    \end{enumerate}

    \subsection{foldr vs foldl}

    I denne oppgaven skal vi se på forskjellen mellom foldr og foldl. Husk at definisjonene til \texttt{foldr} og \texttt{foldl} er som følger.

    \begin{lstlisting}
foldr :: (a -> b -> b) -> b -> [a] -> b
foldr _ b [] = b
foldr f b (a:as) = f a (foldr f b as)

foldl :: (b -> a -> b) -> b -> t a -> b
foldl _ b [] = b
foldl f b (a:as) = foldl f (f b a) as
    \end{lstlisting}



    a) Bruk definisjonen til å regne ut \texttt{foldr (:) [] [1,2,3]}
    
    \begin{ans}
        \hspace{1em}
        \begin{lstlisting}
foldr (:) [] [1,2,3] = (:) 1 foldr (:) [] [2,3]
                       = (:) 1 (:) 2 foldr (:) [] [3]
                       = (:) 1 (:) 2 (:) 3 foldr (:) [] []
                       = (:) 1 (:) 2 (:) 3 []
                       = 1 : 2 : 3 : []
                       = [1,2,3]
        \end{lstlisting}
    \end{ans}
        b) Bruk definisjonen til å regne ut \texttt{foldl }\verb!(\!\texttt{ -> a:l} [] [1,2,3]
    \begin{ans}
        \hspace{1em}
        \begin{lstlisting}
foldl (\l a -> a:l) [] [1,2,3] = foldl (\l a -> a:l) 1:[] [2,3]
                               = foldl (\l a -> a:l) 2:1:[] [3]
                               = foldl (\l a -> a:l) 3:2:1:[] []
                               = 3:2:1:[]
                               = [3,2,1]
        \end{lstlisting}
    \end{ans}
    
    Funksjonen \texttt{repeat :: a -> [a]} er funksjonen som gir en uendelig liste som repeterer et enkelt element.

    \begin{lstlisting}
repeat :: a -> [a]
repeat x = x : repeat x
    \end{lstlisting}

    Funksjonen \texttt{and :: [Bool] -> Bool} kan skrives både ved hjelp av \texttt{foldr} og \texttt{foldl}.

    \texttt{and = foldr (\&\&) True}
    
    eller:

    \texttt{and' = foldl (\&\&) True}

    c) Forklar hva som skjer hvis man evaluerer følgende uttrykk:

    \begin{itemize}
        \item and (repeat False)
        \item and' (repeat False)
    \end{itemize}

    \paragraph{and} - Dersom man tar en liste med bools (\texttt{[False, False, False, False, False]}), vil \texttt{and} gjøre følgende.

    \begin{lstlisting}
and (repeat False) = foldr (&&) True [False, False, False, ..]
                   = True && False foldr (&&) True [False, False, ..]
                   = False foldr (&&) True [False, False, ..]
                   = False foldr (&&) True [False, False, ..]
                   = ...
                   = False
    \end{lstlisting}

    Som vi ser vil resultatet alltid evalueres til \texttt{False}, og derfor vil det returneres \texttt{False}.

    \paragraph{and'} - Hvis vi tar samme liste, vil følgende skje

    \begin{lstlisting}
and' (repeat False) = foldl (&&) True [False, False, False, ..]
                    = foldl (&&) ((&&) True False) [False, False, ..]
                    = ...
    \end{lstlisting}

    Ettersom foldl evalueres lazy, vil denne funksjonen henge for alltid og aldri komme til noe resultat.


\end{document}
