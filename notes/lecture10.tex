\documentclass{article}
\input{preamble.tex}

\begin{document}
    \section{Forelesning 10}

    \subsection{Plan for forelesningen}
    \begin{itemize}
        \item Induktive datastrukturer
        \item AVL trær
        \item Syntakstrær
    \end{itemize}

   \subsection{Induktive data strukturer}
   En datastruktur er induktiv hvis den bruker seg selv som argument til minst en av konstruktørene.

    \begin{lstlisting}
    data binTree a = Empty
        | Branch (BinTree a) a (BinTree a)
    \end{lstlisting}    

    ELementene i en induktiv datastruktur kan settes sammen ved hjelp av konstrukt;rene til store strukturer (trær)

    \subsection{Rekursjon på induktive datastrukturer}
    For å definere en funksjon som tar en indktiv struktur som argumen ved rekursjon

    \begin{itemize}
        \item Gjøre mønster matching på funksjonen:
            \begin{itemize}
                \item Lag et tilfelle for hver konstruktør
                \item Argumentene til konstruktøren blir variabler
            \end{itemize}
        \item Definer funksjonen i hvert tilfelle
        \item Kall funksjonen rekursivt på de variblene som igjen er elementer av den induktive datastrukturen
    \end{itemize}   

    \begin{eg}
        La oss beregne høyden av et binært tre

    \begin{lstlisting}
    height :: BinTree a -> Integer
    height Empty = 0
    height (Branch lefts _ rights)
        = 1 + max (height lefts) (height rights)
    \end{lstlisting}

    Følger denne funksjonen oppskriften fra forrige slide?
    \end{eg}


    \begin{figure}[H]
        \begin{center}
            \includegraphics[width=0.95\textwidth]{rotasjontree}
        \end{center}
        \caption{Illustrasjon av rotasjonsoperasjonen}
        \label{fig:rotasjontree}
    \end{figure}

    \subsection{Rotasjon av trær}
    Et tre kan roteres mot høyre, eller mot venstre, og en rotasjon kan endre høyden til treet.
    \begin{eg}
        Rotasjon av trær
       \begin{figure}[H]
           \begin{center}
               \includegraphics[width=0.65\textwidth]{rotationexample}
           \end{center}
           \caption{Bruk rotasjonsoperasjoner til å balansere treet.}
           \label{fig:rotationexample}
       \end{figure} 

       \begin{figure}[H]
        \centering
        \incfig{rotatetree}
        \end{figure}
    \end{eg}

    \subsection{AVL Trær}
    \begin{itemize}
        \item Introdusert av Adelson-Velsky og Landig i 1962
        \item Består av et binært tre med ekstra balanse informasjon
        \item Definerer et API p3l binære trær som roterer hver gang en ubalanse oppstår
    \end{itemize}

    \subsection{AVL Trær: Implementasjon}
    AVL trær har en balanseinvariant
    \begin{itemize}
        \item Ingen subtrær kan ha en høydeforskjell på mer enn 1.    
        \item En hver forgening er dekorert med høydeforskjellen på de to subtrærne. Denne kalles \textbf{balansefaktoren}.
    \end{itemize}

    \begin{ex}
        Som en øvelse før vi gjør AVL trær, så implementerer vi sorterte lister, dekorert med om de er av odde eller partalls lengde.
    \end{ex}

    \newpage

    \subsection{AVL Datatypen}    
    \begin{lstlisting}
     data BalanceFactor = LeftHeavy
                        | Balanced
                        | RightHeavy
                        deriving (Eq, Show)
    data AVLTree a = Empty
                   | Branch BalanceFactor
                            (AVLTree a)
                            a
                            (AVLTree a)
    \end{lstlisting}
\end{document}
