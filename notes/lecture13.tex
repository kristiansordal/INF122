\documentclass{article}
\input{preamble.tex}

\begin{document}
    \section{Forelesning 13}

    \subsection{Plan for forelesningen}
    \begin{itemize}
        \item \texttt{foldr, foldl, og foldl'}
        \item Eliminatorer
        \item Punktfri notasjon
    \end{itemize}

    \subsection{Fold funksjonen}
    \subsubsection{foldr}

    \begin{lstlisting}
    foldr :: (a -> b -> b) -> b -> [a] -> b
    foldr f z [] = z
    foldr f z (a:as) = f a (foldr f z as)
    \end{lstlisting}

    \begin{eg}
        \textbf{Sum via foldr}

        \[ \texttt{sum = foldr (+) 0} \]

        \begin{itemize}
            \item 0 blir lagt til sist.
            \item Vi må vente til hele listen er brettet ut før vi kan begynne å summere.
        \end{itemize}
        \begin{lstlisting}
        sum [1,2,3] = (foldr (+) 0 1:2:3:[]) 
                     = 1 + (foldr (+) 0 (2:3:[])) 
                     = 1 + (2 + foldr (+) 0 (3:[])) 
                     = 1 + (2 + (3 + foldr (+) 0 []))
                     = 1 + (2 + (3 + 0))
        \end{lstlisting}
    \end{eg}

     \begin{ex}
         Definer følgende funksjoner ved hjelp av \texttt{foldr}
         
         \begin{lstlisting}
         factorial :: (Num a) => a -> a
         head' :: [a] -> Maybe a
         \end{lstlisting}
     \end{ex}

     \subsection{foldl}
     \begin{lstlisting}
     foldl :: (b -> a -> b) -> b -> [a] -> b
     foldl f z [] = z
     foldl f z (a:as) = foldl f (f z a) as
     \end{lstlisting}

     \begin{eg}
         \textbf{Sum via foldl}
         \[ \texttt{sum = foldl (+) 0}\]

        \begin{lstlisting}
        sum [1,2,3] = foldl (+) 0 (1:2:3:[]) 
                     = foldl (+) (0+1) (2:3:[])
                     = foldl (+) ((0+1)+2) (3:[])
                     = foldl (+) (((0+1)+2)+3) []
                     = (((0+1)+2)+3)
                     = 6
        \end{lstlisting}
     \end{eg}

     \begin{itemize}
         \item 0 kommer nå først
         \item Vi kunne begynt å regne ut summen før vi kom til siste linje (tail call recursion)
     \end{itemize}

     MEN: haskell er lazy, og regner derfor ikke ut denne mens vi går

     \subsection{Stack Overflow}

     Med både foldl og foldr er det fare for stack overflow

     \begin{lstlisting}
     foldr (+) 0 [1 .. 10000000000]   
     foldl (+) 0 [1 .. 10000000000]   
     foldr' (+) 0 [1 .. 10000000000]   
     \end{lstlisting}

     \subsection{Eliminatorer}
     \subsubsection{Foldr er spesiell}
     En hver vellfundert rekursjon på lister kan erstattes med \texttt{foldr}

     \begin{eg}
         \textbf{Map via foldr}
         \begin{lstlisting}
         map :: (a -> b) -> [a] -> [b] 
         map _ [] = []
         map f (a:as) =  f a : map f as
         \end{lstlisting}

         Kunne vært

         \begin{lstlisting}
         map :: (a -> b) -> [a] -> [b] 
         map :: f = foldr (\a b -> f a : b) []
         \end{lstlisting}
     \end{eg}

     En slik funksjon som fanger opp vellfundert rekursjon kalles en \textbf{eliminator}.

     Andre datatyper har egne eliminatorer.

     \begin{eg}
        Lister har:

        \[ \texttt{foldr :: (a -> b -> b) -> b -> [a] -> b}\]

        Maybe har:

        \[ \texttt{maybe :: b -> (a -> b) -> Maybe a -> b}  \]

        Either har

        \[ \texttt{either :: (a -> c) -> (b -> c) -> Either a b -> c} \]
     \end{eg}

     \subsection{Punktfri notasjon}

     Det er ikke alltid nødvendig å skrive ut argumentene:

     \[ \texttt{applyFilter fil signal = map fil (iterate tail (extend signal))} \]

     kan skrives som

     \[ \texttt{applyfilter fil = map fil . iterate tail . extend} \]
     \subsection{Unødvendige lambdaer}

     Spesielt nyttig når man lager en funksjon ved hjelp av høyereordens funksjoner

     \begin{lstlisting}
      lessThan3 :: [Integer] -> [Integer]   
      lessThan3 list = filter (\a -> a < 3) list
     \end{lstlisting}

     kunne vært skrevet

     \begin{lstlisting}
      lessThan3 :: [Integer] -> [Integer]   
      lessThan3 list = filter (<3) list
     \end{lstlisting}
\end{document}
