\documentclass{article}
\input{preamble.tex}

\begin{document}
    \section{Forelesning 17}

    \subsection{Pakke og byggesystemer}
    Problem:

    \begin{itemize}
        \item Kompilatoren kompilerer programmet
        \item Hvert program kan avhenge av flere biblioteker
        \item For at programmet skal kompilere må alle bibliotekene som det avhenger av lastes ned og kompileres først.
    \end{itemize}

    Å gjøre denne nedlastingen og kompileringen manuelt er mye arbeid. Et pakke- og byggesystem forsøker å løse dette.

    \subsubsection{Hva er et programbyggesystem?}
    Vanlig ontoligi for et pakke- og bygge system

    \textbf{Pakke:} En helhetlig samling med kildekodefiler som kompileres til et eller flere programmer/bibliotek
\medskip

\textbf{Bygge en pakke: } Kompilere kildekoden og sette sammen programmene/bibliotekent.


\subsubsection{Pakkeversjoner}
Programvare er i stadig utvikling. Derfor kommer pakker i ulike versjoner

\begin{itemize}
    \item Noen ganger er pakke B bare kompatibel med noen versjoner av pakke A.
\end{itemize}

Et pakkesystem må sørge for at det ikke blir konflik når ulike pakker krever ulike versjoner av en avhengighet.

\subsubsection{Eksempler på pakke- og byggesystemer}

Språktilknuttede:

\begin{itemize}
    \item Maven (java)
    \item Pip (Python)
    \item Cabal (Haskell)
    \item npm (JavaScript)
\end{itemize}

Operativsystemtilknyttede:

\begin{itemize}
    \item Make (*nix, språkdiagnostisk)
    \item Ports, portage, nix
    \item Homebrew (MacOS)
\end{itemize}


\subsubsection{Cabal}
Cabal er et pakke- og byggesystem for Haskellprogrammer

\begin{itemize}
    \item En pakke beskrives av en \texttt{.cabal} fil.
    \item Hackage er et sentralt nettsted for offentlige pakker
\end{itemize}

\subsubsection{Cabal Hell}
Cabal har alltid hatt god støtte for å spesifisere hvilke pakker som er kompatible med programmet.
\begin{itemize}
    \item Tidligere var det kun mulig å ha en versjon av en pakke installert.
    \item Det var veldig lett å havne i en situasjon hvor pakker var i konflikt med hverandre.
\end{itemize}

\subsubsection{Stack}
Stack er et alternativt system til cabal-install

\begin{itemize}
    \item Bruker cabal-biblioteket.
    \item Har et annet sentralt pakkeoppsamlingssted
    \item Har en annen strategi for å løse "cabal hell"
    \item Tar også ansvar for GHC installasjonen
\end{itemize}

Stack har et begrep om \textit{snapshot}; en konsistent samling pakker som ikke er i konflikt, sammen med en spesifikk GHC versjon.

\subsection{Testing}
Det finnes mange former for testing:
\begin{itemize}
    \item Unit testing
    \item End-toend testing
    \item Statisk / dynamisk testing
    \item Integrasjonstesting, systemtesting
\end{itemize}

QuickCheck er et verkt;y for property testing.

\subsubsection{Egenskapstesting}
I property testing så tester man at funksjoner har gitte egenskaper
\begin{itemize}
    \item Hver egenskap er en funksjon
    \item Funksjonen tar argumenter og tester om en gitt egenskap holder (Boolsk verdi)
    \item Testene kjøres på tilfeldig generert data
\end{itemize}
\end{document}
