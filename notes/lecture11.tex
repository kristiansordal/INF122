\documentclass{article}
\input{preamble.tex}

\begin{document}
    \section{Forelesning 11}

    \subsection{Plan for forelesningen}
    \begin{itemize}
        \item Insert for AVL Trær
        \item Set og Map
        \item Syntakstrær
    \end{itemize}

    \subsection{Insert i AVL trær}
    Etter insert kan følgende skje
    \begin{itemize}
        \item Treet blir mer balansert
            \begin{itemize}
            \item \texttt{Left / RightHeavy -> Balanced}
            \end{itemize}
        \item Treet blir mindre balansert, men fortsatt tilfredstiller invarianten.
            \begin{itemize}
                \item \texttt{Balanced -> Left / RightHeavy}
            \end{itemize}
        \item Treet blir for ubalansert og trenger rotasjon
    \end{itemize}
De to første tilfellene er lette å håndtere. Det siste tilfellet deler seg
igjen opp i flere tilfeller som må ordnes separat.

\subsubsection{Rebalansering ved rotasjon}
Rebalanseringen er symmetrisk i høyre / venstre tilfeller
Hvis treet var LeftHeavy og venstre subtre har vikst i høyden etter insert, må vi rotere
\begin{itemize}
    \item Hvis venstre subtre er LeftHeavy etter insert kan vi gjøre en enkel rotasjon
    \item Hvis venstre subtre er RightHeavy etter insert må vi gjøre dobbelrotasjon.
\end{itemize}

Kan venstre subtre være Balanced etter insert (gitt at det har vokst)?

\subsection{Syntaks og semantikk}
Syntaks er hvordan vi uttrykker tenkte objekter
\begin{itemize}
    \item 3
    \item \( 2 \cdot 3 - 3 \cdot  3 \)
    \item Idag
    \item 29. september 2022
    \item En verdi i haskell
\end{itemize}

Semantikk er å tilordne betydelse til syntaksen

\begin{eg}
    Vi tilordner "3" samme betydelse som "\( 2 \cdot 6 - 3 \cdot 3 \)".
\end{eg}

Et syntakstre er et trestruktur som representerer et syntaktisk uttrykk.

\begin{eg}
    Syntakstreet til \( 3 \cdot  4 _{ 1 \cdot  2 - 3x} \) er som følger

    \begin{figure}[H]
	\centering
	\incfig{syntaxtree}
\end{figure}
\end{eg}
\end{document}
