\documentclass{article}
\input{preamble.tex}

\begin{document}
    \section{Forelesning 8}
    \subsection{Plan for forelesningen}

    \begin{itemize}
        \item Fortsette med egendefinerte datatyper
        \item Bruke datatyper til å modellere et domene
        \item Trær
    \end{itemize}

    \subsection{Modellering}
    Hvordan modellerer vi domenet vårt i Haskell?
    \begin{itemize}
        \item Vi lager nye datatyper som beskriver ting i domenet vårt
        \item Vi lager funksjoner som lar oss manipulere tingene i domenet vårt
    \end{itemize}

    Hvordan vet vi at programmet vårt er korrekt mtp domenet?
    \begin{itemize}
        \item Sunnhet: Alle velltypede elementer i programmet vårt er gyldie ting i domenet vårt
        \item Kompletthet: Alt vi vil modellere er typede elementer i programmet vårt.
    \end{itemize}

    Det er ikke enten eller, men heller en flytende overgang

    \subsubsection{Invarianter}
    Ofte må vi legge ekstra betingelser på elementene våre for å utelukke ugyldige elementer, verdier og tilstander. Disse egenskapene kalles \textit{invarianter}, fordi de bevares (varierer ikke) av operasjonene i programmet vårt.

    \begin{eg}
        I Obligen modellerer vi:

        \begin{itemize}
            \item Signaler: \texttt{ [Double]}
            \item Filtere på signaler: \texttt{[Double] -> Double} 
        \end{itemize}
    \end{eg}

\end{document}
